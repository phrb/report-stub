%% This is file `elsarticle-template-1-num.tex',
%%
%% Copyright 2009 Elsevier Ltd
%%
%% This file is part of the 'Elsarticle Bundle'.
%% ---------------------------------------------
%%
%% It may be distributed under the conditions of the LaTeX Project Public
%% License, either version 1.2 of this license or (at your option) any
%% later version.  The latest version of this license is in
%%    http://www.latex-project.org/lppl.txt
%% and version 1.2 or later is part of all distributions of LaTeX
%% version 1999/12/01 or later.
%%
%% Template article for Elsevier's document class `elsarticle'
%% with numbered style bibliographic references
%%
%% $Id: elsarticle-template-1-num.tex 149 2009-10-08 05:01:15Z rishi $
%% $URL: http://lenova.river-valley.com/svn/elsbst/trunk/elsarticle-template-1-num.tex $
%%
\documentclass[final,12pt,a4paper]{elsarticle}

%% Use the option review to obtain double line spacing
%% \documentclass[preprint,review,12pt]{elsarticle}

%% Use the options 1p,twocolumn; 3p; 3p,twocolumn; 5p; or 5p,twocolumn
%% for a journal layout:
%% \documentclass[final,1p,times]{elsarticle}
%% \documentclass[final,1p,times,twocolumn]{elsarticle}
%% \documentclass[final,3p,times]{elsarticle}
%% \documentclass[final,3p,times,twocolumn]{elsarticle}
%% \documentclass[final,5p,times]{elsarticle}
%% \documentclass[final,5p,times,twocolumn]{elsarticle}

%% The graphicx package provides the includegraphics command.
\usepackage{graphicx}
%% The amssymb package provides various useful mathematical symbols
\usepackage{amssymb}
%% The amsthm package provides extended theorem environments
%% \usepackage{amsthm}

\usepackage{booktabs}
\usepackage{xcolor}
\usepackage{sourcecodepro}
\usepackage{url}
\usepackage{listings}
\usepackage[utf8]{inputenc}
\usepackage[brazilian]{babel}
\usepackage{multirow}
\usepackage{textcomp}
\usepackage{caption}

\definecolor{Accent}{HTML}{157FFF}

\lstdefinestyle{customMtheme}{%
  backgroundcolor={},
  basicstyle=\ttfamily\scriptsize,
  breakatwhitespace=true,
  breaklines=true,
  captionpos=n,
  commentstyle=\color{orange},
  escapeinside={\%*}{*)},
  extendedchars=true,
  frame=n,
  keywordstyle=\color{Accent},
  language=C++,
  rulecolor=\color{black},
  showspaces=false,
  showstringspaces=false,
  xleftmargin=.5cm,
  xrightmargin=.5cm,
  showtabs=false,
  stepnumber=2,
  stringstyle=\color{gray},
  tabsize=4,
  keywords={void, int, float, main,
  if, else, malloc, NULL,
  fprintf, stderr, for, make, gcc, o, Enter, Ctrl},
  otherkeywords={\#pragma, \#include, \&, \*, +, -, /, [, ], >, <, \$, \., std\=c11}
}
\lstset{basicstyle=\ttfamily\scriptsize,style=customMtheme}

\renewcommand*{\UrlFont}{\ttfamily\scriptsize\relax}

\graphicspath{{./img/}}

%% The lineno packages adds line numbers. Start line numbering with
%% \begin{linenumbers}, end it with \end{linenumbers}. Or switch it on
%% for the whole article with \linenumbers after \end{frontmatter}.
%% \usepackage{lineno}

%% natbib.sty is loaded by default. However, natbib options can be
%% provided with \biboptions{...} command. Following options are
%% valid:

%%   round  -  round parentheses are used (default)
%%   square -  square brackets are used   [option]
%%   curly  -  curly braces are used      {option}
%%   angle  -  angle brackets are used    <option>
%%   semicolon  -  multiple citations separated by semi-colon
%%   colon  - same as semicolon, an earlier confusion
%%   comma  -  separated by comma
%%   numbers-  selects numerical citations
%%   super  -  numerical citations as superscripts
%%   sort   -  sorts multiple citations according to order in ref. list
%%   sort&compress   -  like sort, but also compresses numerical citations
%%   compress - compresses without sorting
%%
%% \biboptions{comma,round}

% \biboptions{}

%% Removing lines when no abstract is given
\makeatletter
\renewcommand{\MaketitleBox}{%
    \resetTitleCounters
        \def\baselinestretch{1}%
        \begin{center}
    \def\baselinestretch{1}%
        \Large \@title \par
        \vskip 18pt
        \normalsize\elsauthors \par
        \vskip 10pt
        \footnotesize \itshape \elsaddress \par
        \end{center}
    \vskip 12pt
}
\makeatother

%% Removing custom footer on fist page
\makeatletter
\def\ps@pprintTitle{%
    \let\@oddhead\@empty
        \let\@evenhead\@empty
        \def\@oddfoot{\centerline{\thepage}%
        }%
    \let\@evenfoot\@oddfoot
}%
\makeatother

\journal{MAC 5742-0219 Introdução à Programação Concorrente, Paralela e Distribuída}

\begin{document}

\begin{frontmatter}

%% Title, authors and addresses

\title{Report Template}

%% use the tnoteref command within \title for footnotes;
%% use the tnotetext command for the associated footnote;
%% use the fnref command within \author or \address for footnotes;
%% use the fntext command for the associated footnote;
%% use the corref command within \author for corresponding author footnotes;
%% use the cortext command for the associated footnote;
%% use the ead command for the email address,
%% and the form \ead[url] for the home page:
%%
%% \title{Title\tnoteref{label1}}
%% \tnotetext[label1]{}
%% \author{Name\corref{cor1}\fnref{label2}}
%% \ead{email address}
%% \ead[url]{home page}
%% \fntext[label2]{}
%% \cortext[cor1]{}
%% \address{Address\fnref{label3}}
%% \fntext[label3]{}


%% use optional labels to link authors explicitly to addresses:
%% \author[label1,label2]{<author name>}
%% \address[label1]{<address>}
%% \address[label2]{<address>}

\author{Pedro Bruel}

\address{Instituto de Matemática e Estatística - Universidade de São Paulo 
(USP) \\ Rua do Matão, 1010 - CEP 05508-090 - São Paulo - SP
}

%%\begin{abstract}
%% Text of abstract
%% Suspendisse potenti. Suspendisse quis sem elit, et mattis nisl. Phasellus
%% consequat erat eu velit rhoncus non pharetra neque auctor. Phasellus eu lacus
%% quam. Ut ipsum dolor, euismod aliquam congue sed, lobortis et orci. Mauris eget
%% velit id arcu ultricies auctor in eget dolor. Pellentesque suscipit adipiscing
%% sem, imperdiet laoreet dolor elementum ut. Mauris condimentum est sed velit
%% lacinia placerat. Vestibulum ante ipsum primis in faucibus orci luctus et
%% ultrices posuere cubilia Curae; Nullam diam metus, pharetra vitae euismod sed,
%% placerat ultrices eros. Aliquam tincidunt dapibus venenatis. In interdum tellus
%% nec justo accumsan aliquam. Nulla sit amet massa augue.
%% \end{abstract}
%%
%% \begin{keyword}
%% Science \sep Publication \sep Complicated
%% keywords here, in the form: keyword \sep keyword

%% MSC codes here, in the form: \MSC code \sep code
%% or \MSC[2008] code \sep code (2000 is the default)

%% \end{keyword}

\end{frontmatter}

%%
%% Start line numbering here if you want
%%
%% \linenumbers

%% main text
\section{Introduction}

%% References
%%
%% Following citation commands can be used in the body text:
%% Usage of \cite is as follows:
%%   \cite{key}          ==>>  [#]
%%   \cite[chap. 2]{key} ==>>  [#, chap. 2]
%%   \citet{key}         ==>>  Author [#]

%% References with bibTeX database:

%% \bibliographystyle{model1-num-names}
%% \bibliography{sample.bib}

%% Authors are advised to submit their bibtex database files. They are
%% requested to list a bibtex style file in the manuscript if they do
%% not want to use model1-num-names.bst.

%% References without bibTeX database:

% \begin{thebibliography}{00}

%% \bibitem must have the following form:
%%   \bibitem{key}...
%%

% \bibitem{}

% \end{thebibliography}


\end{document}

%%
%% End of file `elsarticle-template-1-num.tex'.
